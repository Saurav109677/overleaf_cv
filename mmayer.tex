%%%%%%%%%%%%%%%%%
% This is an example CV created using altacv.cls (v1.6, 21 May 2021) written by
% LianTze Lim (liantze@gmail.com), based on the
% Cv created by BusinessInsider at http://www.businessinsider.my/a-sample-resume-for-marissa-mayer-2016-7/?r=US&IR=T
%
%% It may be distributed and/or modified under the
%% conditions of the LaTeX Project Public License, either version 1.3
%% of this license or (at your option) any later version.
%% The latest version of this license is in
%%    http://www.latex-project.org/lppl.txt
%% and version 1.3 or later is part of all distributions of LaTeX
%% version 2003/12/01 or later.
%%%%%%%%%%%%%%%%

%% Use the "normalphoto" option if you want a normal photo instead of cropped to a circle
% \documentclass[10pt,a4paper,normalphoto]{altacv}

\documentclass[10pt,a4paper,ragged2e,withhyper]{altacv}

%% AltaCV uses the fontawesome5 package.
%% See http://texdoc.net/pkg/fontawesome5 for full list of symbols.

% Change the page layout if you need to
\geometry{left=1.25cm,right=1.25cm,top=1.5cm,bottom=1.5cm,columnsep=1.2cm}

% The paracol package lets you typeset columns of text in parallel
\usepackage{paracol}


% Change the font if you want to, depending on whether
% you're using pdflatex or xelatex/lualatex
\ifxetexorluatex
  % If using xelatex or lualatex:
  \setmainfont{Lato}
\else
  % If using pdflatex:
  \usepackage[default]{lato}
\fi

% Change the colours if you want to
\definecolor{VividPurple}{HTML}{153467}
\definecolor{SlateGrey}{HTML}{2E2E2E}
\definecolor{LightGrey}{HTML}{666666}
% \colorlet{name}{black}
\colorlet{tagline}{VividPurple}
\colorlet{heading}{VividPurple}
\colorlet{headingrule}{VividPurple}
% \colorlet{subheading}{PastelRed}
\colorlet{accent}{VividPurple}
\colorlet{emphasis}{SlateGrey}
\colorlet{body}{LightGrey}

% Change some fonts, if necessary
% \renewcommand{\namefont}{\Huge\rmfamily\bfseries}
% \renewcommand{\personalinfofont}{\footnotesize}
% \renewcommand{\cvsectionfont}{\LARGE\rmfamily\bfseries}
% \renewcommand{\cvsubsectionfont}{\large\bfseries}

% Change the bullets for itemize and rating marker
% for \cvskill if you want to
\renewcommand{\itemmarker}{{\small\textbullet}}
\renewcommand{\ratingmarker}{\faCircle}

%% Use (and optionally edit if necessary) this .tex if you
%% want to use an author-year reference style like APA(6)
%% for your publication list
\input{pubs-authoryear}

%% Use (and optionally edit if necessary) this .tex if you
%% want an originally numerical reference style like IEEE
%% for your publication list
% \input{pubs-num}

%% sample.bib contains your publications
\addbibresource{sample.bib}

\begin{document}
\name{Saurav Kumar}
\tagline{Software Developer}
% Cropped to square from https://en.wikipedia.org/wiki/Marissa_Mayer#/media/File:Marissa_Mayer_May_2014_(cropped).jpg, CC-BY 2.0
%% You can add multiple photos on the left or right
\photoR{2.5cm}{photo}
% \photoL{2cm}{Yacht_High,Suitcase_High}
\personalinfo{%
  % Not all of these are required!
  % You can add your own with \printinfo{symbol}{detail}
  \email{saurav109677@gmail.com}
  \phone{91-6295135550}
%   \mailaddress{Patna City, Patna, India - 800008}
  \location{Patna}
%   \homepage{marissamayr.tumblr.com}
%   \twitter{@marissamayer}
%   \linkedin{saurav109677}
  \github{github.com/saurav109677} % I'm just making this up though.
%   \orcid{0000-0000-0000-0000} % Obviously making this up too.
  %% You can add your own arbitrary detail with
  %% \printinfo{symbol}{detail}[optional hyperlink prefix]
  % \printinfo{\faPaw}{Hey ho!}
  %% Or you can declare your own field with
  %% \NewInfoFiled{fieldname}{symbol}[optional hyperlink prefix] and use it:
  % \NewInfoField{gitlab}{\faGitlab}[https://gitlab.com/]
  % \gitlab{your_id}
	%%
  %% For services and platforms like Mastodon where there isn't a
  %% straightforward relation between the user ID/nickname and the hyperlink,
  %% you can use \printinfo directly e.g.
  % \printinfo{\faMastodon}{@username@instace}[https://instance.url/@username]
  %% But if you absolutely want to create new dedicated info fields for
  %% such platforms, then use \NewInfoField* with a star:
  % \NewInfoField*{mastodon}{\faMastodon}
  %% then you can use \mastodon, with TWO arguments where the 2nd argument is
  %% the full hyperlink.
  % \mastodon{@username@instance}{https://instance.url/@username}
}

\makecvheader

%% Depending on your tastes, you may want to make fonts of itemize environments slightly smaller
\AtBeginEnvironment{itemize}{\small}

%% Set the left/right column width ratio to 6:4.
\columnratio{0.6}

% Start a 2-column paracol. Both the left and right columns will automatically
% break across pages if things get too long.
\begin{paracol}{2}

\cvsection{Experience}

\cvevent{Software Developer}{TCS}{July 2019 -- Ongoing}{Pune, Maharashtra}
\begin{itemize}
\item On-site working experience with one of renowned organization
\item Learned technology like iBM-iSeries
% \item Built Yahoo's mobile, video and social businesses from nothing in 2011 to \$1.6 billion in GAAP revenue in 2015
% \item Tripled the company's mobile base to over 600 million monthly active users and generated over \$1 billion of mobile advertising revenue last year
\end{itemize}

\divider

\cvsection{IT Skills}

\cvskill{Java}{4.5}
\cvskill{ReactJS}{4}
\cvskill{NodeJS}{3.5}
\cvskill{AWS Cloud}{3}
% \cvskill{Git}{8}
% \cvskill{Latex}{9}

\divider

\cvsection{Projects}

\textbf{Live Youth Forum App}\\\vspace{0.9mm}
	\begin{itemize}[leftmargin=*, noitemsep]
	\item See the app working :-  https://youth-forum-app.heroku.com
	\item ReactJS | NodeJS | MongoDB | RestAPI
	\item Realtime app - working for one organization to get the follow-up reports of students for whom they educate.
	\end{itemize}
	
\divider

\textbf{Web scrapping app}\\\vspace{0.9mm}
	\begin{itemize}[leftmargin=*, noitemsep]
	\item NodeJS | Pupeteer
	\item Automate websites having no-captcha using NodeJs Pupeteer Module
	\item In this, I have automated several coding challenge on website and let the program solve the problem.
	\end{itemize}
	
\divider

\textbf{AWS hosted website}\\\vspace{0.9mm}
	\begin{itemize}[leftmargin=*, noitemsep]
	\item S3 | CodeDeploy | EC2 | Cloud9
	\item Build and hosted many localhost websites to AWS cloud infrastructure and manage the traffic.
	\end{itemize}

\divider

\cvsection{Languages}

\cvskill{English}{5}
% \divider

\cvskill{Hindi}{5}
\divider

\cvsection{Education}

\cvevent{B Tech.\ in Computer Science}{Haldia Institute of Technology}{Aug 2015 -- June 2019}{}

\divider




% use ONLY \newpage if you want to force a page break for
% ONLY the currentc column


%% Switch to the right column. This will now automatically move to the second
%% page if the content is too long.
\switchcolumn

\cvsection{Life Philosophy}
\begin{quote}
``If you don't have any shadows, you're not standing in the light.''
\end{quote}

\divider

\cvsection{Most Proud of}

\cvachievement{\faTrophy}{Courage I had}{to take a sinking ship and try to make it float}

\divider

% \cvachievement{\faHeartbeat}{Persistence \& Loyalty}{I showed despite the hard moments and my willingness to stay with Yahoo after the acquisition}

% \divider

\cvachievement{\faChartLine}{Learning Curve}{always want to learn new technologies}

\divider

\cvachievement{\faStar}{Leadership}{Inspires team to serve the organisation together}

\divider

\cvsection{Strengths}

\cvtag{Hard-working (18/24)}
\cvtag{Determined}\\
\cvtag{Motivator \& Leader}

\divider\smallskip

\cvtag{Web Development}
\cvtag{JSP}
\cvtag{ReactJS}
\cvtag{Java}
\cvtag{AWS}
\cvtag{HTML}
\cvtag{CSS}
\cvtag{API}
\cvtag{Jenkins}
\cvtag{CI/CD Pipeline}
\cvtag{Redux}
\cvtag{RPGLE}
\cvtag{CLLE}
\cvtag{IBM iSeries}

\divider


\cvsection{Hobbies}

\cvtag{Reading Spiritual Books}
\cvtag{Computer Tricks}
\cvtag{Playing Musical Instruments}\\



% \cvevent{B.S.\ in Symbolic Systems}{Stanford University}{Sept 1993 -- June 1997}{}



\end{paracol}

\end{document}
